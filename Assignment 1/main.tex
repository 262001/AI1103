\documentclass[journal,12pt,two column]{IEEEtran}
\usepackage[cmex10]{amsmath}
\usepackage[utf8]{inputenc}
\setlength{\parindent}{0pt}


\title{AI1103 - Assignment 1}
\author{Monika Kharadi - CS20BTECH11026}
\date{March 2021}

\begin{document}

\maketitle

\section*{\large\textbf{Problem}}

2.12. In a school, there are 1000 students, out of which 430 are girls. It is known that out of 430, 10 percent of girls study in class XII. What is the probability that a student chosen randomly studies in Class XII given that the chosen student is a girl ?
{\section*{\textbf{Solution}}}
Total number of girls are 430 out of 1000 students \\
Total number of girls in Class XII : 10 \% of total girls 
\begin{align}
=\frac{10*430}{100} 
\end{align}
\begin{align}
= 43    
\end{align}
Let X $\in$ $\{0,1\}$ be the random variable such that 1 represents girl, 0 represents boy.
\begin{align}
P(X=1)= \frac{430}{1000}   
\end{align}
Let Y $\in$ $\{0,1\}$ be the random variable such that 1 represents chosen student is in Class XII, 0 represents chosen student is not in Class XII.
\begin{align}
P(Y=1)= \frac{1}{2}   
\end{align}
\begin{align}
P(Y=0)= \frac{1}{2}   
\end{align}
Now,\\
Chosen student is a girl in Class XII
\begin{align}    
P(Y=1\vert X=1)=\frac{43}{1000} 
\end{align}
Chosen student is a girl not in Class XII
\begin{align}
P(X=1\vert Y=0)=\frac{387}{1000}
\end{align}

\begin{align}    
=\frac{P(X=1\vert Y=1)\cdot P(Y=1)}{\sum_{i=0}^{1} P(X=1\vert Y=i)\cdot P(Y=i)}
\end{align}
\begin{align}
=\frac{P(X=1 \vert Y=1)\cdot P(Y=1)}{\splitfrac{P(X=1\vert Y=0)P(Y=0)}{+P(X=1\vert Y=1)P(Y=1)}}
\end{align}
\begin{align}
=\frac{0.043\cdot (\frac{1}{2})}{0.387\cdot (\frac{1}{2}) + 0.043\cdot (\frac{1}{2})} 
\end{align}
\begin{align}
=\frac{0.043}{0.43}    
\end{align}
\begin{align}
=\frac{1}{10}    
\end{align}
Hence, the probability that a student chosen randomly studies in Class XII given that the chosen student is a girl is 0.1.





\end{document}
